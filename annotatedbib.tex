
\documentclass[11pt]{article}

\usepackage{fullpage}

\usepackage[style=verbose]{biblatex}
\bibliography{hfoss}

\begin{document}

\centerline{\large\bf HFOSS Annotated Bibliography}

\vspace{.25in}



\begin{itemize}
\item \cite{Ellis:2012}

\begin{quote}
Free and Open Source Software (FOSS) offers a transparent development environment and community in which to involve students. Students can learn much about software development and professionalism by contributing to an on-going project. However, the number of FOSS projects is very large and there is a wide range of size, complexity, domains, and communities, making selection of an ideal project for students difficult. This paper addresses the need for guidance when selecting a FOSS project for student involvement by presenting an approach for FOSS project selection based on clearly identified criteria. The approach is based on several years of experience involving students in FOSS projects.
\end{quote}

\item \cite{Ellis:2013}

\begin{quote}
Many faculty members are excited by the learning potential inherent in student participation in a Free and Open Source Software (FOSS) project. Student learning can range from software development to technical writing to team skills to professionalism and more. The altruistic nature of humanitarian FOSS provides additional appeal to students by providing the ability to do some social good. However, selection of an appropriate project can be difficult due to the large number of humanitarian FOSS projects available, and the wide range of size, complexity, domains, and communities in those projects. We have developed an approach to FOSS project selection [1] based on several years of experience involving students in humanitarian FOSS projects. This workshop will provide participants with a hands-on experience in selecting such a project. Participants will understand the key aspects of FOSS projects that are important when evaluating a project for use in the classroom. Participants will also be guided through the process of identifying and evaluating candidate projects for their classes.
\end{quote}

\item \cite{Ellis:2014}

\begin{quote}
Many computing degree programs emphasize real-world experiences for students. One way of achieving this experience is via student participation in a Free and Open Source Software (FOSS) project. FOSS projects provide a wide range of learning opportunities that spans software development to technical writing to team skills to professionalism and more. The altruistic nature of humanitarian FOSS (HFOSS) projects provides additional appeal to students by providing the ability to do some social good. HOFSS projects have been shown to motivate students and provide professional and software engineering learning [1].
\end{quote}

\item \cite{Ellis:2014a}

\begin{quote}
This paper describes student learning within the environment of an HFOSS project that is jointly shared between the GNOME Accessibility Team and three academic institutions. This effort differs from many project-based learning efforts in that the project is shared between the academic participants and the HFOSS community. By involving students in an HFOSS project, learning is started via apprenticeship which allows students to learn from professionals while preparing them for their professional life. Learning within the community of an ongoing FOSS project guides students in the first steps towards understanding the importance of life-long learning as well as providing an initial understanding of the ways in which such learning occurs. The results of a student survey and observations of student reflection papers are discussed.
\end{quote}

\item \cite{Kussmaul:2012}

\begin{quote}
Many faculty members (and students) desire to know more about free \& open source software (FOSS) development and its tools and practices. This workshop introduces participants to collaboration tools \& techniques used in FOSS. In particular, we will focus on task tracking systems and version control systems, which are unfamiliar to many faculty and students. To help participants understand what these tools do and how to use them, we will use process oriented guided inquiry learning (POGIL) activities. In POGIL, learners work in groups of 3 or 4 in guided activities that are structured to help them construct new knowledge. In the two hands-on activities, teams will work through a series of increasingly sophisticated models. In each model, teams will use tools, answer questions, explore options, and report out their findings and lessons learned. We particularly welcome students, who should enjoy the activities and could help faculty observe the strengths and limitations of the tools and activities. Participants will receive copies of all activities, presentation slides, and other materials, and an annotated bibliography on FOSS, POGIL, and related topics.
\end{quote}

\item \cite{Ellis:2013a}

\begin{quote}
Studies have shown that the "near peer" experience where students of various levels are jointly involved in co-learning activities can motivate students and support wide learning. Humanitarian Free and Open Source Software (HFOSS) projects have shown promise for educating students using real-world projects within a global, professional community. Leveraging the near peer experience within an HFOSS project allows beginning students to get earlier exposure to large, complex systems while providing the more advanced students the opportunity to practice communication, coordination, and leadership skills. This poster describes initial steps towards the development of an HFOSS project by a mixed team of students of various levels and from three different institutions..
\end{quote}

\item \cite{Ellis:2013b}

\begin{quote}
Many faculty members are excited by the learning potential inherent in student participation in a Free and Open Source Software (FOSS) project. Student learning can range from software development to technical writing to team skills to professionalism and more. The altruistic nature of humanitarian FOSS provides additional appeal to students by providing the ability to do some social good. However, selection of an appropriate project can be difficult due to the large number of humanitarian FOSS projects available, and the wide range of size, complexity, domains, and communities in those projects. We have developed an approach to FOSS project selection [1] based on several years of experience involving students in humanitarian FOSS projects. This workshop will provide participants with a hands-on experience in selecting such a project. Participants will understand the key aspects of FOSS projects that are important when evaluating a project for use in the classroom. Participants will also be guided through the process of identifying and evaluating candidate projects for their classes.
\end{quote}

\end{itemize}

\end{document}

